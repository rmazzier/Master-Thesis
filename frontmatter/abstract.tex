%!TEX root = ../dissertation.tex

Research on Deep Learning has achieved remarkable results in recent years, mainly thanks to the computing power of modern computers and the increasing availability of large data sets. However, deep neural models are universally considered black boxes: they employ sub-symbolic representations of knowledge, which are inherently opaque to human beings trying to derive explanations. In this work, we first give a survey on the research field of Explainable AI, providing more rigorous definitions of the concepts of interpretability and explainability.
We then delve deeper in the research field of Neural Symbolic Integration, which tackles the task of integrating the statistical learning power of machine learning with the symbolic and abstract world of logic. Specifically, we analyze Knowledge Enhanced Neural Networks (KENN) \cite{daniele2019kenn}, a special kind of residual layer for neural architectures which makes it possible to inject symbolic logical knowledge inside a neural network. We describe and analyze experimental results on relational data on the task of collective classification, and study how KENN is able to automatically learn the importance of logical rules from the training data. We finally review explainability methods for KENN, proposing ways to extract explanations for the predictions provided by the model.